% mnras_template.tex 
%
% LaTeX template for creating an MNRAS paper
%
% v3.3 released April 2024
% (version numbers match those of mnras.cls)
%
% Copyright (C) Royal Astronomical Society 2015
% Authors:
% Keith T. Smith (Royal Astronomical Society)

% Change log
%
% v3.3 April 2024
%   Updated \pubyear to print the current year automatically
% v3.2 July 2023
%	Updated guidance on use of amssymb package
% v3.0 May 2015
%    Renamed to match the new package name
%    Version number matches mnras.cls
%    A few minor tweaks to wording
% v1.0 September 2013
%    Beta testing only - never publicly released
%    First version: a simple (ish) template for creating an MNRAS paper

%%%%%%%%%%%%%%%%%%%%%%%%%%%%%%%%%%%%%%%%%%%%%%%%%%
% Basic setup. Most papers should leave these options alone.
\documentclass[fleqn,usenatbib]{mnras}

% MNRAS is set in Times font. If you don't have this installed (most LaTeX
% installations will be fine) or prefer the old Computer Modern fonts, comment
% out the following line
\usepackage{newtxtext,newtxmath}
% Depending on your LaTeX fonts installation, you might get better results with one of these:
%\usepackage{mathptmx}
%\usepackage{txfonts}

% Use vector fonts, so it zooms properly in on-screen viewing software
% Don't change these lines unless you know what you are doing
\usepackage[T1]{fontenc}

% Allow "Thomas van Noord" and "Simon de Laguarde" and alike to be sorted by "N" and "L" etc. in the bibliography.
% Write the name in the bibliography as "\VAN{Noord}{Van}{van} Noord, Thomas"
\DeclareRobustCommand{\VAN}[3]{#2}
\let\VANthebibliography\thebibliography
\def\thebibliography{\DeclareRobustCommand{\VAN}[3]{##3}\VANthebibliography}


%%%%% AUTHORS - PLACE YOUR OWN PACKAGES HERE %%%%%

% Only include extra packages if you really need them. Avoid using amssymb if newtxmath is enabled, as these packages can cause conflicts. newtxmatch covers the same math symbols while producing a consistent Times New Roman font. Common packages are:
\usepackage{graphicx}	% Including figure files
\usepackage{amsmath}	% Advanced maths commands

%%%%%%%%%%%%%%%%%%%%%%%%%%%%%%%%%%%%%%%%%%%%%%%%%%

%%%%% AUTHORS - PLACE YOUR OWN COMMANDS HERE %%%%%

% Please keep new commands to a minimum, and use \newcommand not \def to avoid
% overwriting existing commands. Example:
%\newcommand{\pcm}{\,cm$^{-2}$}	% per cm-squared

%%%%%%%%%%%%%%%%%%%%%%%%%%%%%%%%%%%%%%%%%%%%%%%%%%

%%%%%%%%%%%%%%%%%%% TITLE PAGE %%%%%%%%%%%%%%%%%%%

% Title of the paper, and the short title which is used in the headers.
% Keep the title short and informative.
\title{The Motion of the Dark Matter Halo Remnant of the MW-M31 Merger: Prograde or Retrograde?}

% The list of authors, and the short list which is used in the headers.
% If you need two or more lines of authors, add an extra line using \newauthor
\author{Savannah Smith}

\date{Due date: 11 April 2025}

% Prints the current year, for the copyright statements etc. To achieve a fixed year, replace the expression with a number. 
\pubyear{\the\year{}}

% Don't change these lines
\begin{document}
\label{firstpage}
\pagerange{\pageref{firstpage}--\pageref{lastpage}}
\maketitle

% Select between one and six entries from the list of approved keywords.
% Don't make up new ones.
\begin{keywords}
Cold Dark Matter Theory -- Dark Matter Halo -- Stellar Disk -- Virial Radius -- Hernquist Profile
\end{keywords}

%%%%%%%%%%%%%%%%%%%%%%%%%%%%%%%%%%%%%%%%%%%%%%%%%%

%%%%%%%%%%%%%%%%% BODY OF PAPER %%%%%%%%%%%%%%%%%%

\section{Introduction}

\paragraph{} Galaxies and dark matter halos evolve together through large-scale mergers, with the dark matter halo and the baryonic matter merging on different timescales. \textbf{Cold Dark Matter Theory} is the theory that dark matter consists of weakly interacting massive particles (WIMPs) with relatively small thermal velocities, allowing them to form small structures of less than one solar mass \citep{Diemand+2011}. It is now theorized that CDM combines to create a cosmic web structure, connecting everything in the observable universe. In addition to the cosmic web, we can define \textbf{Dark Matter (DM) Halos} as a spherical region of higher-density dark matter that surrounds a galaxy \citep{Drakos+2019}. DM halos cannot be physically observed using any electromagnetic radiation, but we can observe the gravitational effects on satellite bodies, since the halos extend past the visible baryonic matter of a galaxy, such as the disk. The \textbf{stellar disk} is a relatively thin and flat component of a galaxy that contains the majority of a galaxy's stars and stellar material (baryonic material). We can analyze both the DM halo and the stellar disk using their respective important radii. For the baryon disk, the \textbf{virial radius} is the radius where the average density of the galaxy inside that radius is approximately 200 times higher than the critical density of the universe \citep{Salucci+2007}. For the DM halo, the \textbf{Hernquist profile} describes the density distribution of the DM within the halo and produces the scale radius, which represents the radius at which the DM halo goes from high density to lower density \citep{Hern+1990, Dubinski+1999}. Then we can roughly define where the density of the stellar disk and the DM halo decrease significantly and how certain properties, such as angular momentum, change at different radii throughout a major merger. Because of their size and the large amount of mass that results in stronger gravitational attraction, the DM halos will be the first to interact in a major merger, as well as be the first to interact with other satellite bodies. Therefore, mergers, both minor and major, greatly affect the structure and shape of DM halos \citep{Drakos+2019}.

\paragraph{} Galaxy mergers play a crucial role in galaxy evolution. While galaxies will evolve on their own, mergers allow for a large amount of interaction between two massive astronomical objects. The term \textbf{galaxy} is derived from the Greek word for "milky". \textbf{Galaxy} is defined as a "gravitationally bound collection of stars whose properties cannot be explained by a combination of baryons and Newton's laws of gravity" \citep{Willman+2012}, although \cite{Willman+2012} does discuss other ways to identify and classify galaxies, such as stellar kinematics and [Fe/H] spread. This definition implies a dependence on dark matter, without explicitly saying so, as \textbf{Cold Dark Matter (CDM) Theory} is not considered a physical law due to the lack of observational evidence. The term \textbf{galaxy evolution} refers to the formation of a galaxy and how that galaxy changes over time. Because CDM halos account for the majority of a galaxy's mass, it follows that the merging of the DM halos would greatly influence the nature of the galaxy remnant.

\paragraph{} From \citep{Drakos+2019} we know from simulation data that the DM halos take a considerable amount of time for their shape to stabilize, see figure \ref{fig:DrakosMerger}. From the same figure, we see that the peak circular velocity remains relatively constant, but the radius at which that circular velocity peaks varies throughout the merger \citep{Drakos+2019}. Multiple studies have found that the stellar matter that lies within the disk of a galaxy directly influences the shape of the dark matter halo \citep{Prada+2019}. In simulations where baryons are present, the DM halo is much more spherical than in simulations where only DM is present \citep{Chua+2019}. Although some studies claim that the orientations of dark matter halos would remain relatively constant through mergers, \cite{Baptista+2023} claims, through an analysis with the LMC, that a major merger would result in an alteration in the halos' orientation due to the change in angular momentum. Also, instead of DM halos being supported by rotation, \cite{Diemand+2011} claims that DM halos are instead supported by \textit{almost} isotropic velocity dispersions, but further claims that there is approximately the same amount of positive and negative angular momentum material relative to any reference frame \citep{Diemand+2011}. In addition, \cite{Diemand+2011} argues that at radii greater than 10\% of the virial radius, the stellar disk orientation and the DM halo shape are uncorrelated.

\begin{figure}              
    \centering
    \includegraphics[width=0.6\columnwidth]{Drakos+2019fig7.png}
    \caption{This figure assumes a triaxial halo where \( a \), \( b \), and \( c \) represent the major, median, and minor axes, respectively. From the top plot to the bottom plot, the figure analyzes the axis ratios \( c/a \) and \( c/b \), as well as the parameters \( r'_{\text{peak}} \) and \( v'_{\text{peak}} \) with respect to their original, pre-merger values of \( r_{\text{peak}} \) (radius at which the circular velocity peaks) and \( v_{\text{peak}} \) (the peak circular velocity), respectively. All of these variables are plotted with respect to time, and each \( N \) represents a different simulation resolution. The vertical dotted lines show the time by which the two DM halos had completely merged. Although the peak circular velocity remains relatively constant, the other parameters take time to stabilize \citep{Drakos+2019}.}
    \label{fig:DrakosMerger}
    
\end{figure}

\paragraph{} There are many open questions concerning the evolution of galaxies and dark matter halos through mergers, and these questions are actively being studied. A first major question is how the angular momentum of the DM halo evolves throughout a major merger and if the effects remain after a considerable timescale. Further, it is still in question whether or not the spin of the DM halo remnant favors one galaxy over the other \citep{Rod+2017}. A second major question outlined in \cite{Drakos+2019} is how we can compare a galaxy's substructure to relate its past merger history to its final state. \cite{Drakos+2019} discusses many broad conclusions for size, shape, and spin, but notes that more research would be beneficial for clarification. A third major question considering mergers is how the overall mass of the galaxy remnant compares to the mass of each of the host galaxies, due to the fact that large amounts of matter, both dark matter and stellar, will be ejected during the merger. Researchers are trying to solve these open questions by using simulation data of various galaxies with various density profiles. Some simulations isolate the DM halo, and some use data for both a stellar disk \textit{and} a DM halo. 


\section{This Project}

\paragraph{} In this paper, we will study whether the DM halo remnant of the Milky Way (MW) and Andromeda (M31) merger will be prograde or retrograde with respect to the rotation of the stellar disk. Prograde motion refers to motion that exists in the \textit{same} direction as the object it is surrounding, whereas retrograde motion refers to motion that exists in the \textit{opposite} direction as the object it is surrounding. In the case of this paper, I will be analyzing the angular momentum of both the DM halo and the stellar disk of both MW and M31 pre-merger and the MW-M31 remnant post-merger. We can compare the direction of the angular momenta in three-dimensional space and determine if there is prograde or retrograde motion occurring. By examining the angular momentum across different times, we can analyze how the motion changes through the MW-M31 merger.

\paragraph{} Of the open questions previously discussed, this project addresses the first major question about how the angular momentum of the DM halo evolves throughout a major merger. For determining if the DM halo's motion is prograde or retrograde, it is essential that we analyze the angular momentum of both the stellar disk and the DM halo.

\paragraph{} This question is important for galaxy evolution because DM halos contain a large amount of a galaxy's total mass, even though it is not able to be directly studied using electromagnetic radiation. Because mergers are not an astronomical event that happens on a regular basis within an observable radius, we use simulations of galaxy mergers, both minor and major, to determine the changes that occur throughout a merger. Throughout these mergers, angular momentum can oppose each other as the galaxies collide and cause changes to the stellar disk. A change to the DM halo means a change to the environment that surrounds the other main components of a galaxy (disk and bulge), and so a dramatic change in the halo could result in a change in the baryonic components.


\section{Methodology}

\paragraph{} This project uses simulation data outlined in \cite{Marel+2012}, which discusses the "future dynamical evolution of the system composed of the MW, M31, and M33" \citep{Marel+2012} by using \textit{N}-body simulations and semi-analytic orbit integrations for each of these galaxies. \textit{N}-body simulation refers to simulations that consider a large number of particles and determine the position and velocity vector of each particle, considering their interactions with each other. This simulation considered only stellar material and dark matter within the local group system (no gas particles). The DM halo was represented using a Hernquist density profile and, for the stellar disk, the virial radius was used to determine many properties such as the virial mass and the concentration \citep{Marel+2012}.

\paragraph{} To determine whether the DM halo of the galaxy remnant has prograde or retrograde motion with respect to the stellar disk, we must calculate the angular momentum of both the DM halo and the stellar disk. To do so, we will be using both the stellar disk (type 2) and the DM halo (type 1) particles for the MW and M31 from the simulation data. The resolution is not a crucial part of these calculations, so the low-resolution data will be used. For each particle type, we can calculate the angular momentum and then take the dot product of the disk angular momentum and the halo angular momentum. If the dot product is negative, the orbit of the dark matter halo is prograde. If the dot product is positive, the orbit of the dark matter halo is retrograde. We can plot the sign of the dot product over an array of times (snapshots of the simulation data) to analyze how the rotation of the DM halo evolves throughout the MW-M31 merger. In relation to this plot, one similar to figure \ref{ChenProRetro} could be created to show the difference in the angular momenta of each particle at different radii surrounding the stellar disk. This would require looking at a two-dimensional slice of the three-dimensional simulation data.

\begin{figure}
    \centering
    \includegraphics[width=\columnwidth]{Chen+2022fig1.png}
    \caption{This figure represents the flow pattern of a surrounding body (i.e., DM halo) around an embedded companion (i.e., stellar disk). The color represents the specific angular momentum $J_0$ relative to the companion. Red represents prograde motion and blue represents retrograde motion. The upper row shows motion at perigee and the lower row shows motion at apogee \citep{Chen+2022}.}
    \label{ChenProRetro}
    
\end{figure}

\paragraph{} Firstly, we must rotate the M31 simulation because the data is not represented edge-on as the MW data is. Then, we can calculate the angular momentum of each galaxy component using the following equation for angular momentum, represented as $\vec{L_i}$:
\begin{center}
    $\vec{L} = \sum_{i} \vec{r_i} \times \vec{p_i} = \sum_{i} m_i (\vec{r_i} \times \vec{v_i})$
\end{center}
Where the sum is over every particle in the simulation data. $m_i$, $\vec{r_i}$, and $\vec{v_i}$ represent the mass, position vector (x, y, and z components), and velocity vector (x, y, and z components) of each individual particle. We can import the mass of every particle directly from the simulation data, but the position and velocity vectors need to be adjusted to be in the frame of the center of mass of the galaxy. We can then apply a mask to the position vectors to account for a set amount of either the stellar disk particles or the DM halo particles so that we are not including particles that are too far away from the merger.
\\To determine if the dark matter halo is prograde or retrograde, it is necessary to compute the dot product of the angular momenta of the dark matter halo and the baryon disk. For this, we can say the following by definition of the dot product:
\begin{center}
    $\vec{L_{halo}} \cdot \vec{L_{disk}} = |\vec{L_{halo}}||\vec{L_{disk}}|cos\theta$ $\rightarrow$ $cos\theta = \frac{\vec{L_{halo}} \cdot \vec{L_{disk}}}{|\vec{L_{halo}}||\vec{L_{disk}}|}$
\end{center}
The information that will express whether the orbit around the disk is prograde or retrograde is the cosine expression. If the cosine term is negative, the orbit of the dark matter halo is prograde. If the cosine term is positive, the orbit of the dark matter halo is retrograde. We can also analyze how this prograde or retrograde motion changes throughout the merger. To do so, we can choose an array of snapshots to collect simulation data from and perform the above calculations for every snapshot. We can also analyze how the prograde and retrograde motion changes at certain radii. For this, we can choose an array of radii within the scale radius for the DM halo to see how the halo is affected by the stellar disk.

\paragraph{} The first plot that I will create will be the cosine term from the dot product of the angular momenta versus time. There will be two lines: one representing the MW and the other representing M31. This will demonstrate how the direction of the angular momentum of each galaxy changes with time throughout the merger. The second plot that I create will be analyzing how the halo motion (the dot product of the angular momenta) varies with radius. This plot will be similar to figure \ref{ChenProRetro} in the sense that it will show how the motion of a surrounding body is prograde or retrograde at certain radii. While it would be ideal to create a plot that is similar in nature to those in \citep{Chen+2022}, if we are unable to produce a plot that shows the angular momenta of every particle, it is possible to create a plot with the following design. There will be four lines: one representing the MW pre-merger, another representing M31 pre-merger, another representing the MW-M31 remnant shortly after the merger, and another representing the MW-M31 remnant after the merger when the components are more stable. This plot will demonstrate if the dot product is changing at every chosen radius or if it is relatively constant.

\paragraph{} After watching video simulations of the local group, it seems that relative to each other, the Milky Way and M31 are rotating in opposite directions. It also seems that the Milky Way has both a prograde and a retrograde component to the dark matter halo rotation, while the dark matter halo of M31 seems to have a prograde orbit \citep{Deason+2011}. Because of this, I would hypothesize that the dark matter halo remnant post-collision would orbit retrograde to the baryon disk remnant post-collision. This is due to the fact that as an entire galaxy system, the individual components of each galaxy are rotating in the same direction, but the MW and M31 are rotating in opposite directions with respect to each other. Because the halo has a lower density with the mass more spread out among the halo, I would assume that the disk of the remnant will have more angular momentum than the halo of the remnant because the baryonic matter orbiting within the disk will be moving at greater velocities than the objects within the halo. Considering this, it may be the case that the angular momenta of the disks may affect the angular momenta of the halos throughout the merger. I would also hypothesize that as radius increases from the galactic center, the direction of the angular momentum would not remain constant shortly after the merger, due to the chaos. I do think that as the remnant has time to stabilize, the prograde or retrograde motion of the DM halo will be constant over the scale radius from the Hernquist profile. 



%%%%%%%%%%%%%%%%%%%% REFERENCES %%%%%%%%%%%%%%%%%%

% The best way to enter references is to use BibTeX:

\bibliographystyle{mnras}
\bibliography{RA4} % if your bibtex file is called example.bib


% Alternatively you could enter them by hand, like this:
% This method is tedious and prone to error if you have lots of references
%\begin{thebibliography}{99}
%\bibitem[\protect\citeauthoryear{Author}{2012}]{Author2012}
%Author A.~N., 2013, Journal of Improbable Astronomy, 1, 1
%\bibitem[\protect\citeauthoryear{Others}{2013}]{Others2013}
%Others S., 2012, Journal of Interesting Stuff, 17, 198
%\end{thebibliography}

%%%%%%%%%%%%%%%%%%%%%%%%%%%%%%%%%%%%%%%%%%%%%%%%%%


% Don't change these lines
\bsp	% typesetting comment
\label{lastpage}
\end{document}

% End of mnras_template.tex
